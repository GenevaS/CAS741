\documentclass{article}

\usepackage{tabularx}
\usepackage{booktabs}

\usepackage[backend=bibtex,style=ieee]{biblatex}
\bibliography{ProblemStatement_Refs}

\title{CAS 741: Problem Statement\\Companion Cube Calculator (CCC)}

\author{Geneva Smith (\texttt{smithgm})}

\date{\today}

%% Comments

\usepackage{color}

\newif\ifcomments\commentstrue

\ifcomments
\newcommand{\authornote}[3]{\textcolor{#1}{[#3 ---#2]}}
\newcommand{\todo}[1]{\textcolor{red}{[TODO: #1]}}
\else
\newcommand{\authornote}[3]{}
\newcommand{\todo}[1]{}
\fi

\newcommand{\wss}[1]{\authornote{blue}{SS}{#1}}
\newcommand{\an}[1]{\authornote{magenta}{Author}{#1}}

\begin{document}

\maketitle

\begin{table}[hp]
\caption{Revision History} \label{TblRevisionHistory}
\begin{tabularx}{\textwidth}{llX}
\toprule
\textbf{Date} & \textbf{Developer(s)} & \textbf{Change}\\
\midrule
September 15, 2017 & Geneva Smith & Added an additional environment constraint; 
clarified the purpose of the project \\
September 14, 2017 & Geneva Smith & First draft \\
\bottomrule
\end{tabularx}
\end{table}

The GLaDOS architecture\cite{glados} is a specialized game engine that enables 
game designers to create Non-Player Characters (NPCs) that react to their 
environment by using models of emotion from psychology. A key component of the 
architecture is the primary appraisal module, where inputs from the environment 
are converted into emotion values. The equations currently used for this task 
are not well informed by any scientific research, and will be incrementally 
revised as more information is collected. These new equations are required to 
conform to the same mathematical range as the currently implemented equations. 
Ensuring that this requirement is maintained over consecutive iterations of the 
equations will become tedious and time consuming, distracting from the main 
task of developing more rigorous and explainable emotion equations. To help 
streamline this task, I propose the development of the Companion Cube 
Calculator (CCC).

The CCC is a tool for calculating the range of a mathematical equation based on 
its input variables. Users will be able to specify input domains for variables 
that are known, and the tool will determine the domains for unspecified input 
variables. For example, the equation $y = x + 5$ should return $y \in (-\inf, 
\inf), x \in (-\inf, \inf)$ whereas the constrained equation of $y = x + 5, x 
\geq 0$ should return $y \in [5, \inf)$. For the initial version of the CCC 
tool, the possible values will be limited to the domain of real numbers 
($\Re$). In the special case of piece-wise equations, the ideal tool would be 
able to accept the full equation set as a single input. However, if this is 
beyond the scope of the initial version, piece-wise equations can be entered as 
individual equations with constrained input domains. This project can be 
implemented as a command line tool. However, one of the only ways to ensure 
that there are no user input errors is to create a Graphical User Interface 
(GUI) that will only allow syntactically correct inputs. If this requirement is 
implemented, the initial version of the CCC tool will only be available for 
Windows operating systems.

%Computerized models of creature behaviours is a type of artificial 
%intelligence 
%(AI) that is frequently used by game designers to define  in their games. 
%While many academic approaches to AI are typically too 
%resource intensive or exact for this application, the methodical and 
%research-based approach to defining behaviour equations can be used to create 
%the desired effects. This also enables the benefits related to testing and 
%maintainability that such an approach offers. Even though this project is 
%targeted at game designers, it can be used for any project that requires the 
%development of mathematical equations.

\printbibliography

\end{document}